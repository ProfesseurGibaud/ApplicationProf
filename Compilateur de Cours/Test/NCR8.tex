\textit{Réponse : }\\

Supposons par l'absurde que $\sqrt{2}$ soit rationnel. Donc il existe $a,b \in \Z$, avec $b \neq 0$.\\
On suppose de plus que $a$ et $b$ sont premiers entre eux (en effet en les divisant par leur PGCD on obtient une fraction de nombres premiers). 
Donc au moins $a$ ou $b$ est impair, en effet si ils étaient tous les deux pairs, on pourrait simplifier la fraction par 2.  
On a : 
\[
\sqrt{(2)} = \frac{a}{b} \ \text{ donc } 2 = \frac{a^2}{b^2} \ \text{ donc } 2 b^2 = a^2
\]

Donc $a^2$ est pair et par conséquent $a$ est pair. On écrit alors : $a = 2k$ avec $k \in \Z$.
On obtient : 
\begin{align*}
	2 = \frac{a^2}{b^2} \\
	2 = \frac{(2k)^2}{b^2}\\
	2 b^2 = 4k^2 \\
	b^2 = 2 k^2
\end{align*}

Donc $b$ est aussi pair, ce qui contredit le fait que $a$ et $b$ sont premiers entre eux.