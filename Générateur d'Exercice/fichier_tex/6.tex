\documentclass{article}
 \usepackage[utf8]{inputenc}
  \title{Devoir de Rattrapage}
  \date{A rendre pour le Mardi 7 Novembre 8h}\usepackage{natbib} 
 \usepackage{graphicx} 
 \usepackage{amsmath} 
 \usepackage{mathtools}\begin{document}\maketitle
 \section{Exercices de In\'equation Faciles}

 R\'esoudre les in\'equations suivantes : 
In\'equation num\'ero 0 \[3 x^{2} \leq \left(x + 8\right)^{2}\]In\'equation num\'ero 1 \[\left(4 x + 4\right)^{2} \geq \left(9 x + 7\right)^{2}\]In\'equation num\'ero 2 \[\left(3 x + 1\right)^{2} \geq 4 x + 3\]In\'equation num\'ero 3 \[\left(x + 1\right)^{2} \geq 9 x^{2} + 6 x + 1\]In\'equation num\'ero 4 \[9 x^{2} + x + 9 \geq \left(5 x + 9\right)^{2}\]In\'equation num\'ero 5 \[\left(7 x + 10\right)^{2} \geq 7 x^{2} + 3 x + 8\]In\'equation num\'ero 6 \[3 x + 7 \geq x + 1\]In\'equation num\'ero 7 \[\left(5 x + 7\right)^{2} \geq 4 x + 9\]In\'equation num\'ero 8 \[9 x^{2} + 7 x + 1 \geq \left(3 x + 6\right)^{2}\]In\'equation num\'ero 9 \[\left(2 x + 8\right)^{2} \geq 5 x^{2} + 8 x\]In\'equation num\'ero 10 \[\left(9 x + 9\right)^{2} \geq 9 x^{2} + 8 x + 1\]In\'equation num\'ero 11 \[7 x^{2} + 2 \geq \left(2 x + 7\right)^{2}\]In\'equation num\'ero 12 \[x + 1 \leq 5 x^{2} + 8 x + 1\]In\'equation num\'ero 13 \[6 x + 5 \geq \left(3 x + 7\right)^{2}\]In\'equation num\'ero 14 \[9 x + 4 \geq \left(8 x + 2\right)^{2}\]In\'equation num\'ero 15 \[9 x^{2} + 2 x + 3 \leq \left(x + 6\right)^{2}\]In\'equation num\'ero 16 \[6 x^{2} + x + 7 \leq 4 x + 2\]In\'equation num\'ero 17 \[\left(7 x + 9\right)^{2} \leq \left(x + 8\right)^{2}\]In\'equation num\'ero 18 \[5 x + 5 \geq 7 x^{2} + 3 x + 6\]In\'equation num\'ero 19 \[7 x + 2 \leq \left(7 x + 10\right)^{2}\]
 \section{Exercices de In\'equation Moyens}

 R\'esoudre les in\'equations suivantes : 
In\'equation num\'ero 0 \[\frac{3 x + 2}{3 x + 7} \geq \frac{\sqrt{3}}{6}\]In\'equation num\'ero 1 \[\frac{5 x + 6}{8 x + 3} \geq \frac{4 \sqrt{5} \pi}{5}\]In\'equation num\'ero 2 \[\frac{x + 2}{3 x + 1} \leq \frac{\sqrt{6} \pi}{3}\]In\'equation num\'ero 3 \[\frac{8 x + 6}{8 x + 2} \leq \sqrt{15}\]In\'equation num\'ero 4 \[\frac{8 x + 10}{4 x + 1} \geq \frac{\sqrt{6}}{4}\]In\'equation num\'ero 5 \[\frac{x + 2}{8 x + 3} \geq 2 \sqrt{5} \pi\]In\'equation num\'ero 6 \[\frac{2 x + 10}{9 x + 5} \geq \frac{\sqrt{2}}{3 \pi}\]In\'equation num\'ero 7 \[\frac{10 x + 7}{x + 10} \geq \frac{\sqrt{143}}{11 \pi}\]In\'equation num\'ero 8 \[\frac{9 x + 5}{2 x + 3} \leq \sqrt{5}\]In\'equation num\'ero 9 \[\frac{6 x + 2}{9 x + 5} \geq \frac{\sqrt{10} \pi}{5}\]In\'equation num\'ero 10 \[\frac{4 x + 10}{5 x + 10} \geq \frac{3 \sqrt{22} \pi}{11}\]In\'equation num\'ero 11 \[\frac{7 x + 5}{2 x + 5} \leq \frac{\sqrt{2}}{2}\]In\'equation num\'ero 12 \[\frac{5 x + 10}{4 x + 2} \leq \frac{\sqrt{11}}{11 \pi}\]In\'equation num\'ero 13 \[\frac{6 x + 2}{2 x + 4} \leq \frac{\pi}{2}\]In\'equation num\'ero 14 \[\frac{4 x + 8}{10 x + 1} \leq \frac{2 \sqrt{65}}{13}\]In\'equation num\'ero 15 \[\frac{6 x + 9}{5 x + 2} \geq \frac{\sqrt{210}}{15}\]In\'equation num\'ero 16 \[\frac{6 x + 9}{5 x + 8} \leq \sqrt{2}\]In\'equation num\'ero 17 \[\frac{2 x + 2}{10 x + 9} \geq \frac{\sqrt{5}}{5}\]In\'equation num\'ero 18 \[\frac{4 x + 1}{10 x + 9} \geq \sqrt{17}\]In\'equation num\'ero 19 \[\frac{8 x + 2}{8 x + 5} \geq \frac{\sqrt{13}}{13}\]In\'equation num\'ero 20 \[\frac{6 x + 4}{2 x + 8} \leq 1.0\]In\'equation num\'ero 21 \[\frac{2 x + 6}{10 x + 6} \geq \frac{1}{3 \pi}\]In\'equation num\'ero 22 \[\frac{3 x + 5}{8 x + 5} \geq \frac{2 \sqrt{34}}{17}\]In\'equation num\'ero 23 \[\frac{5 x + 10}{2 x + 4} \leq \frac{\sqrt{238}}{14}\]In\'equation num\'ero 24 \[\frac{2 x + 5}{2 x + 4} \geq \frac{\sqrt{70}}{7}\]In\'equation num\'ero 25 \[\frac{2 x + 3}{7 x + 9} \geq \frac{\sqrt{238} \pi}{14}\]In\'equation num\'ero 26 \[\frac{2 x + 6}{9 x + 1} \leq \frac{1}{4 \pi}\]In\'equation num\'ero 27 \[\frac{4 x + 2}{6 x + 8} \geq \pi\]In\'equation num\'ero 28 \[\frac{x + 9}{9 x + 5} \leq \frac{\sqrt{13} \pi}{3}\]In\'equation num\'ero 29 \[\frac{x + 6}{x + 3} \geq 2 \sqrt{2}\]In\'equation num\'ero 30 \[\frac{7 x + 3}{4 x + 8} \leq \frac{\sqrt{21}}{6 \pi}\]In\'equation num\'ero 31 \[\frac{7 x + 2}{6 x + 1} \geq \frac{\sqrt{11}}{4 \pi}\]In\'equation num\'ero 32 \[\frac{4 x + 7}{x + 6} \geq \frac{\sqrt{209}}{11 \pi}\]In\'equation num\'ero 33 \[\frac{5 x + 9}{6 x + 6} \leq \sqrt{3}\]In\'equation num\'ero 34 \[\frac{8 x + 7}{10 x + 4} \leq \sqrt{3}\]In\'equation num\'ero 35 \[\frac{10 x + 4}{9 x + 8} \geq \frac{\sqrt{55}}{11}\]In\'equation num\'ero 36 \[\frac{9 x + 8}{x + 4} \geq \frac{\sqrt{6}}{3}\]In\'equation num\'ero 37 \[\frac{4 x + 3}{x + 2} \leq \frac{\sqrt{5}}{5}\]In\'equation num\'ero 38 \[\frac{4 x + 10}{8 x + 4} \geq \frac{\sqrt{13}}{2 \pi}\]In\'equation num\'ero 39 \[\frac{2 x + 6}{10 x + 4} \leq \frac{1}{3 \pi}\]
 \section{Exercices de In\'equation Durs}

 R\'esoudre les in\'equations suivantes : 
In\'equation num\'ero 0 \[\frac{x^{2} - 4 x + 4}{3 x + 10} \leq \frac{x^{2} - 2 x + 1}{3 x + 10}\]In\'equation num\'ero 1 \[\frac{9 x^{2} - 6 x + 1}{9 x + 9} \leq \frac{9 x^{2} + 30 x + 25}{9 x + 9}\]In\'equation num\'ero 2 \[\frac{7 x + 10}{7 x + 1} \leq \frac{6 x + 1}{9 x + 7}\]In\'equation num\'ero 3 \[\frac{25 x^{2} - 50 x + 25}{5 x + 2} \leq \frac{25 x^{2} - 10 x + 1}{5 x + 2}\]In\'equation num\'ero 4 \[\frac{7 x + 5}{5 x + 4} \geq \frac{5 x + 1}{4 x + 1}\]In\'equation num\'ero 5 \[\frac{x + 2}{x + 5} \geq \frac{4 x + 4}{x + 7}\]In\'equation num\'ero 6 \[\frac{7 x + 9}{6 x + 2} \geq \frac{3 x + 5}{5 x + 9}\]In\'equation num\'ero 7 \[\frac{8 x + 2}{x + 8} \geq \frac{6 x + 10}{7 x + 4}\]In\'equation num\'ero 8 \[\frac{x + 5}{6 x + 7} \geq \frac{3 x + 2}{3 x + 3}\]In\'equation num\'ero 9 \[\frac{4 x + 10}{6 x + 7} \geq \frac{10 x + 3}{2 x + 1}\]In\'equation num\'ero 10 \[\frac{x + 3}{9 x + 9} \leq \frac{9 x + 5}{x + 1}\]In\'equation num\'ero 11 \[\frac{5 x + 8}{x + 9} \geq \frac{6 x + 7}{4 x + 1}\]In\'equation num\'ero 12 \[\frac{x + 10}{8 x + 7} \geq \frac{9 x + 8}{7 x + 4}\]In\'equation num\'ero 13 \[\frac{9 x^{2} - 24 x + 16}{3 x + 1} \geq \frac{4 x^{2} - 16 x + 16}{3 x + 1}\]In\'equation num\'ero 14 \[\frac{3 x + 8}{4 x + 4} \leq \frac{9 x + 1}{7 x + 5}\]In\'equation num\'ero 15 \[\frac{x^{2} - 10 x + 25}{9 x + 5} \leq \frac{16 x^{2} - 32 x + 16}{9 x + 5}\]In\'equation num\'ero 16 \[\frac{2 x + 2}{3 x + 6} \geq \frac{9 x + 6}{x + 9}\]In\'equation num\'ero 17 \[\frac{x^{2} - 2 x + 1}{6 x + 3} \leq \frac{4 x^{2} - 12 x + 9}{6 x + 3}\]In\'equation num\'ero 18 \[\frac{2 x + 2}{5 x + 1} \leq \frac{x + 6}{9 x + 4}\]In\'equation num\'ero 19 \[\frac{3 x + 10}{9 x + 3} \geq \frac{4 x + 9}{4 x + 4}\]In\'equation num\'ero 20 \[\frac{4 x + 4}{x + 3} \leq \frac{6 x + 3}{2 x + 9}\]In\'equation num\'ero 21 \[\frac{4 x^{2} + 20 x + 25}{2 x + 1} \geq \frac{16 x^{2} - 24 x + 9}{2 x + 1}\]In\'equation num\'ero 22 \[\frac{16 x^{2} + 8 x + 1}{7 x + 5} \leq \frac{9 x^{2} - 18 x + 9}{7 x + 5}\]In\'equation num\'ero 23 \[\frac{4 x^{2} - 12 x + 9}{5 x + 1} \leq \frac{4 x^{2} - 8 x + 4}{5 x + 1}\]In\'equation num\'ero 24 \[\frac{2 x + 2}{6 x + 10} \leq \frac{3 x + 9}{x + 1}\]In\'equation num\'ero 25 \[\frac{9 x^{2} - 6 x + 1}{3 x + 8} \geq \frac{16 x^{2} - 24 x + 9}{3 x + 8}\]In\'equation num\'ero 26 \[\frac{4 x^{2} - 20 x + 25}{3 x + 9} \geq \frac{25 x^{2} - 10 x + 1}{3 x + 9}\]In\'equation num\'ero 27 \[\frac{4 x^{2} - 12 x + 9}{9 x + 5} \geq \frac{25 x^{2} - 20 x + 4}{9 x + 5}\]In\'equation num\'ero 28 \[\frac{25 x^{2} - 30 x + 9}{8 x + 7} \leq \frac{x^{2} + 10 x + 25}{8 x + 7}\]In\'equation num\'ero 29 \[\frac{3 x + 1}{8 x + 5} \geq \frac{7 x + 9}{5 x + 2}\]In\'equation num\'ero 30 \[\frac{9 x^{2} - 6 x + 1}{8 x + 9} \leq \frac{x^{2} - 8 x + 16}{8 x + 9}\]In\'equation num\'ero 31 \[\frac{9 x + 10}{2 x + 6} \leq \frac{10 x + 4}{5 x + 3}\]In\'equation num\'ero 32 \[\frac{6 x + 10}{x + 4} \leq \frac{2 x + 8}{x + 10}\]In\'equation num\'ero 33 \[\frac{8 x + 3}{9 x + 2} \leq \frac{6 x + 10}{8 x + 8}\]In\'equation num\'ero 34 \[\frac{5 x + 8}{7 x + 6} \geq \frac{x + 10}{6 x + 8}\]In\'equation num\'ero 35 \[\frac{x + 9}{6 x + 2} \geq \frac{x + 6}{10 x + 1}\]In\'equation num\'ero 36 \[\frac{x^{2} - 6 x + 9}{6 x + 5} \geq \frac{16 x^{2} - 8 x + 1}{6 x + 5}\]In\'equation num\'ero 37 \[\frac{3 x + 7}{8 x + 3} \geq \frac{x + 3}{x + 5}\]In\'equation num\'ero 38 \[\frac{16 x^{2} - 16 x + 4}{7 x + 9} \leq \frac{x^{2} - 2 x + 1}{7 x + 9}\]In\'equation num\'ero 39 \[\frac{9 x^{2} + 12 x + 4}{7 x + 1} \leq \frac{25 x^{2} - 50 x + 25}{7 x + 1}\]
 \section{Exercices de Tableaux de Variation Faciles}

 Donner les variations des fonctions suivantes : 
Tableaux de Variation num\'ero 0 \[f(x) = \sqrt{13} x^{2} + x + \pi x + 2\]Tableaux de Variation num\'ero 1 \[f(x) = \sqrt{14} \pi x^{3} + 7 \sqrt{2} \pi x^{2} + 8 \sqrt{5} x + 9 \sqrt{2} \pi\]Tableaux de Variation num\'ero 2 \[f(x) = \sqrt{13} x^{2} + 7 x + \sqrt{11} \pi x + \sqrt{13} \pi\]Tableaux de Variation num\'ero 3 \[f(x) = \left(3 x + 2\right) \left(9 x + 3\right)^{2}\]Tableaux de Variation num\'ero 4 \[f(x) = x^{2} + 3 x + 2 \sqrt{5} \pi x + \sqrt{5}\]Tableaux de Variation num\'ero 5 \[f(x) = 4 \sqrt{2} x^{3} + 7 \sqrt{13} \pi x^{2} + 6 \pi x + 18\]Tableaux de Variation num\'ero 6 \[f(x) = \pi x^{2} + \sqrt{14} x + 9 x + 2 \sqrt{3} + 4\]Tableaux de Variation num\'ero 7 \[f(x) = 3 x + 2 \sqrt{5} x + \sqrt{7} \pi x + 7\]Tableaux de Variation num\'ero 8 \[f(x) = \left(4 x + 10\right)^{2} \cdot \left(10 x + 7\right)\]Tableaux de Variation num\'ero 9 \[f(x) = \left(2 \pi x + 2 \pi\right)^{3}\]Tableaux de Variation num\'ero 10 \[f(x) = \sqrt{7} x^{2} + \sqrt{5} \pi x + 10 x + 5 + 4 \pi\]Tableaux de Variation num\'ero 11 \[f(x) = 2 \sqrt{2} x^{3} + 8 \sqrt{3} x^{2} + 32 x + 5 \sqrt{5} \pi\]Tableaux de Variation num\'ero 12 \[f(x) = \left(x + 4\right)^{3}\]Tableaux de Variation num\'ero 13 \[f(x) = \left(7 x + 6\right) \left(8 x + 8\right)^{2}\]Tableaux de Variation num\'ero 14 \[f(x) = \pi x^{2} + x + 2 \sqrt{3} x + 5 + 2 \sqrt{2} \pi\]Tableaux de Variation num\'ero 15 \[f(x) = 10 \sqrt{5} x^{3} + 14 \sqrt{3} \pi x^{2} + 9 \sqrt{2} x + 7 \sqrt{2}\]Tableaux de Variation num\'ero 16 \[f(x) = x^{3} + 6 \sqrt{17} x^{2} + 8 \sqrt{3} x + 1\]Tableaux de Variation num\'ero 17 \[f(x) = 12 \sqrt{3} \pi x^{3} + 9 \pi x^{2} + 2 \sqrt{5} \pi x + 6 \sqrt{3}\]Tableaux de Variation num\'ero 18 \[f(x) = \left(8 x + 5\right) \left(9 x^{2} + 4 x + 4\right)\]Tableaux de Variation num\'ero 19 \[f(x) = \left(6 \sqrt{7} x + 9 \sqrt{2} \pi\right)^{3}\]
 \section{Exercices de Tableaux de Variation Moyens}

 Donner les variations des fonctions suivantes : 
Tableaux de Variation num\'ero 0 \[f(x) = \frac{\sqrt{19} x + 3 \sqrt{2}}{\pi x + 3}\]Tableaux de Variation num\'ero 1 \[f(x) = \frac{\sqrt{15} \pi x + \sqrt{2}}{\sqrt{5} x + \sqrt{17} \pi}\]Tableaux de Variation num\'ero 2 \[f(x) = \frac{x + \pi}{\sqrt{10} \pi x + \sqrt{13}}\]Tableaux de Variation num\'ero 3 \[f(x) = \frac{\sqrt{5} \pi x + 2 \sqrt{5} \pi}{\sqrt{10} x + \sqrt{14} \pi}\]Tableaux de Variation num\'ero 4 \[f(x) = \frac{x + \sqrt{10}}{\sqrt{11} x + \sqrt{5} \pi}\]Tableaux de Variation num\'ero 5 \[f(x) = \frac{2 \pi x + \sqrt{3}}{\sqrt{15} \pi x + \sqrt{7} \pi}\]Tableaux de Variation num\'ero 6 \[f(x) = \frac{\sqrt{15} \pi x + 1}{\pi x + \sqrt{17}}\]Tableaux de Variation num\'ero 7 \[f(x) = \left(6 x + 3\right) \sqrt{10 x + 7}\]Tableaux de Variation num\'ero 8 \[f(x) = \frac{\sqrt{17} \pi x + \sqrt{17}}{x + 2 \sqrt{2}}\]Tableaux de Variation num\'ero 9 \[f(x) = \sqrt{x + 6} \cdot \left(2 x + 3\right)\]Tableaux de Variation num\'ero 10 \[f(x) = \frac{\sqrt{19} x + \pi}{x + 1}\]Tableaux de Variation num\'ero 11 \[f(x) = \sqrt{2 x + 3} \cdot \left(2 x + 7\right)\]Tableaux de Variation num\'ero 12 \[f(x) = \sqrt{x + 2} \cdot \left(2 x + 5\right)\]Tableaux de Variation num\'ero 13 \[f(x) = \sqrt{3 x + 8} \cdot \left(8 x + 1\right)\]Tableaux de Variation num\'ero 14 \[f(x) = \sqrt{2 x + 2} \cdot \left(4 x + 6\right)\]Tableaux de Variation num\'ero 15 \[f(x) = \sqrt{x + 7} \cdot \left(9 x + 7\right)\]Tableaux de Variation num\'ero 16 \[f(x) = \frac{\sqrt{19} \pi x + \pi}{x + \pi}\]Tableaux de Variation num\'ero 17 \[f(x) = \frac{\sqrt{19} \pi x + \sqrt{6}}{2 \sqrt{5} x + 1}\]Tableaux de Variation num\'ero 18 \[f(x) = \frac{\sqrt{17} x + \sqrt{11} \pi}{x + 1}\]Tableaux de Variation num\'ero 19 \[f(x) = \frac{\sqrt{5} x + 3 \pi}{\pi x + \sqrt{7}}\]Tableaux de Variation num\'ero 20 \[f(x) = \sqrt{9 x + 2} \cdot \left(9 x + 3\right)\]Tableaux de Variation num\'ero 21 \[f(x) = \frac{\sqrt{14} x + 2 \pi}{2 x + 2 \sqrt{2}}\]Tableaux de Variation num\'ero 22 \[f(x) = \frac{\pi x + \pi}{\sqrt{17} x + 4}\]Tableaux de Variation num\'ero 23 \[f(x) = \left(2 x + 10\right) \sqrt{4 x + 8}\]Tableaux de Variation num\'ero 24 \[f(x) = \sqrt{2 x + 7} \cdot \left(3 x + 2\right)\]Tableaux de Variation num\'ero 25 \[f(x) = \frac{\pi x + \sqrt{2} \pi}{\sqrt{13} x + 3 \sqrt{2} \pi}\]Tableaux de Variation num\'ero 26 \[f(x) = \frac{\pi x + \sqrt{10}}{2 \sqrt{3} x + \sqrt{19} \pi}\]Tableaux de Variation num\'ero 27 \[f(x) = \frac{2 \sqrt{5} x + \sqrt{14}}{x + 1}\]Tableaux de Variation num\'ero 28 \[f(x) = \sqrt{6 x + 8} \cdot \left(9 x + 6\right)\]Tableaux de Variation num\'ero 29 \[f(x) = \frac{\sqrt{14} x + \sqrt{6}}{\sqrt{6} x + 2 \sqrt{3}}\]Tableaux de Variation num\'ero 30 \[f(x) = \frac{2 \pi x + \sqrt{19} \pi}{\sqrt{14} \pi x + \sqrt{7}}\]Tableaux de Variation num\'ero 31 \[f(x) = \left(4 x + 3\right) \sqrt{10 x + 8}\]Tableaux de Variation num\'ero 32 \[f(x) = \frac{x + \sqrt{3}}{\sqrt{5} x + 2 \sqrt{5}}\]Tableaux de Variation num\'ero 33 \[f(x) = \frac{\sqrt{2} \pi x + 1}{2 \pi x + 2 \sqrt{3} \pi}\]Tableaux de Variation num\'ero 34 \[f(x) = \sqrt{8 x + 5} \cdot \left(9 x + 2\right)\]Tableaux de Variation num\'ero 35 \[f(x) = \left(6 x + 4\right) \sqrt{8 x + 7}\]Tableaux de Variation num\'ero 36 \[f(x) = \sqrt{x + 6} \cdot \left(8 x + 1\right)\]Tableaux de Variation num\'ero 37 \[f(x) = \frac{\pi x + \sqrt{13}}{\pi x + \sqrt{14}}\]Tableaux de Variation num\'ero 38 \[f(x) = \sqrt{x + 1} \cdot \left(5 x + 2\right)\]Tableaux de Variation num\'ero 39 \[f(x) = \frac{\sqrt{15} x + \sqrt{10}}{2 \sqrt{2} x + \sqrt{3}}\]
 \section{Exercices de Tableaux de Variation Durs}

 Donner les variations des fonctions suivantes : 
Tableaux de Variation num\'ero 0 \[f(x) = 2 \sqrt{3} \pi \left(\sqrt{14} x + 1\right) \sqrt{\left(7 x + 3\right)^{2}}\]Tableaux de Variation num\'ero 1 \[f(x) = \pi \sqrt{5 x + 9} \left(\pi x + 3\right)\]Tableaux de Variation num\'ero 2 \[f(x) = \sqrt{7} \left(\sqrt{2} \pi x + \sqrt{17} \pi\right) \sqrt{\left(5 x + 8\right)^{2}}\]Tableaux de Variation num\'ero 3 \[f(x) = 2 \pi \left(\sqrt{10} x + 3 \sqrt{2}\right) \sqrt{\left(9 x + 6\right)^{2}}\]Tableaux de Variation num\'ero 4 \[f(x) = \sqrt{15} \pi \left(\sqrt{19} \pi x + \pi\right) \sqrt{\left(5 x + 8\right)^{2}}\]Tableaux de Variation num\'ero 5 \[f(x) = \sqrt{10} \left(\sqrt{14} x + \sqrt{19}\right) \sqrt{\left(3 x + 7\right)^{2}}\]Tableaux de Variation num\'ero 6 \[f(x) = \frac{\pi \sqrt{\sqrt{11} x + 2 \sqrt{2} \pi}}{7 x^{2} + 4 x + 3}\]Tableaux de Variation num\'ero 7 \[f(x) = 2 \left(\pi x + 1\right) \sqrt{\left(5 x + 7\right)^{2}}\]Tableaux de Variation num\'ero 8 \[f(x) = \frac{\sqrt{17} \pi \sqrt{\sqrt{3} \pi x + \pi}}{\left(10 x + 6\right)^{2}}\]Tableaux de Variation num\'ero 9 \[f(x) = \frac{\sqrt{x + \sqrt{2} \pi}}{x^{2} + 4}\]Tableaux de Variation num\'ero 10 \[f(x) = \frac{3 \sqrt{2} \sqrt{2 \sqrt{3} x + \sqrt{10} \pi}}{\left(9 x + 5\right)^{2}}\]Tableaux de Variation num\'ero 11 \[f(x) = \frac{\sqrt{17} \sqrt{\sqrt{3} \pi x + \sqrt{5}}}{9 x^{2} + 3 x + 9}\]Tableaux de Variation num\'ero 12 \[f(x) = \frac{\sqrt{17} \sqrt{\sqrt{19} \pi x + \sqrt{17}}}{7 x^{2} + 6 x + 1}\]Tableaux de Variation num\'ero 13 \[f(x) = \frac{\sqrt{19} \pi \sqrt{2 \pi x + \sqrt{7} \pi}}{4 x^{2} + 9 x + 1}\]Tableaux de Variation num\'ero 14 \[f(x) = \left(\sqrt{11} \pi x + 2 \sqrt{3}\right) \sqrt{\left(x + 1\right)^{2}}\]Tableaux de Variation num\'ero 15 \[f(x) = \sqrt{6} \pi \left(\sqrt{17} \pi x + 3 \pi\right) \sqrt{x^{2} + 7 x + 2}\]Tableaux de Variation num\'ero 16 \[f(x) = \frac{\sqrt{\sqrt{19} \pi x + \sqrt{6} \pi}}{4 x^{2} + 8 x + 4}\]Tableaux de Variation num\'ero 17 \[f(x) = \left(x + \pi\right) \sqrt{\left(10 x + 8\right)^{2}}\]Tableaux de Variation num\'ero 18 \[f(x) = 2 \sqrt{3} \cdot \left(2 x + \sqrt{10}\right) \sqrt{\left(7 x + 3\right)^{2}}\]Tableaux de Variation num\'ero 19 \[f(x) = \sqrt{2} \pi \left(\sqrt{17} x + \sqrt{6} \pi\right) \sqrt{\left(8 x + 9\right)^{2}}\]Tableaux de Variation num\'ero 20 \[f(x) = \frac{\sqrt{\sqrt{7} x + 2 \sqrt{2}}}{7 x^{2} + 8 x + 6}\]Tableaux de Variation num\'ero 21 \[f(x) = \sqrt{3} \left(\sqrt{17} \pi x + 1\right) \sqrt{\left(7 x + 7\right)^{2}}\]Tableaux de Variation num\'ero 22 \[f(x) = \frac{4 \pi \sqrt{\sqrt{7} \pi x + 1}}{\left(5 x + 2\right)^{2}}\]Tableaux de Variation num\'ero 23 \[f(x) = \frac{\sqrt{13} \pi \sqrt{3 x + 1}}{\left(3 x + 5\right)^{2}}\]Tableaux de Variation num\'ero 24 \[f(x) = \sqrt{5} \left(\pi x + 1\right) \sqrt{8 x^{2} + 7 x + 6}\]Tableaux de Variation num\'ero 25 \[f(x) = \frac{\sqrt{11} \pi \sqrt{3 \sqrt{2} \pi x + \pi}}{8 x^{2} + 3}\]Tableaux de Variation num\'ero 26 \[f(x) = \sqrt{2} \left(\sqrt{15} x + 2 \sqrt{3}\right) \sqrt{6 x^{2} + 7 x + 6}\]Tableaux de Variation num\'ero 27 \[f(x) = 3 \sqrt{2} \left(\pi x + \sqrt{10} \pi\right) \sqrt{\left(5 x + 3\right)^{2}}\]Tableaux de Variation num\'ero 28 \[f(x) = \frac{\sqrt{17} \pi \sqrt{2 x + \sqrt{13}}}{\left(9 x + 6\right)^{2}}\]Tableaux de Variation num\'ero 29 \[f(x) = \sqrt{13} \left(x + 2 \sqrt{3} \pi\right) \sqrt{\left(4 x + 5\right)^{2}}\]Tableaux de Variation num\'ero 30 \[f(x) = \frac{\sqrt{3} \sqrt{3 \sqrt{2} \pi x + 3 \sqrt{2} \pi}}{\left(4 x + 9\right)^{2}}\]Tableaux de Variation num\'ero 31 \[f(x) = \frac{3 \sqrt{2} \pi \sqrt{2 \pi x + \sqrt{7}}}{\left(9 x + 1\right)^{2}}\]Tableaux de Variation num\'ero 32 \[f(x) = \frac{\sqrt{17} \sqrt{\sqrt{7} \pi x + \sqrt{7}}}{\left(9 x + 2\right)^{2}}\]Tableaux de Variation num\'ero 33 \[f(x) = \frac{\pi \sqrt{\sqrt{2} \pi x + \sqrt{17} \pi}}{2 x + 8}\]Tableaux de Variation num\'ero 34 \[f(x) = \frac{3 \pi \sqrt{\sqrt{10} \pi x + 4 \pi}}{6 x^{2} + 3 x + 1}\]Tableaux de Variation num\'ero 35 \[f(x) = \sqrt{6} \cdot \left(3 \pi x + \sqrt{10}\right) \sqrt{\left(8 x + 5\right)^{2}}\]Tableaux de Variation num\'ero 36 \[f(x) = \frac{\sqrt{5} \pi \sqrt{\sqrt{7} x + \sqrt{7}}}{\left(5 x + 6\right)^{2}}\]Tableaux de Variation num\'ero 37 \[f(x) = \frac{2 \pi \sqrt{\sqrt{15} \pi x + \sqrt{10}}}{4 x + 2}\]Tableaux de Variation num\'ero 38 \[f(x) = \frac{\sqrt{x + 3}}{\left(7 x + 2\right)^{2}}\]Tableaux de Variation num\'ero 39 \[f(x) = \sqrt{15} \sqrt{x + 5} \cdot \left(4 x + \sqrt{19}\right)\]\end{document}