\documentclass{article}
 \usepackage[utf8]{inputenc}
  \title{Devoir de Rattrapage}
  \date{A rendre pour le Mardi 7 Novembre 8h}\usepackage{natbib} 
 \usepackage{graphicx} 
 \usepackage{amsmath} 
 \usepackage{mathtools}\begin{document}\maketitle
 \section{Exercices de In\'equation Faciles}

 R\'esoudre les in\'equations suivantes : 
In\'equation num\'ero 0 \[\left(4 x + 2\right)^{2} \geq x^{2} + x + 1\]In\'equation num\'ero 1 \[3 x^{2} + 9 x + 9 \leq 6 x^{2} + 9 x + 3\]In\'equation num\'ero 2 \[8 x + 8 \geq 2 x^{2} + 3 x + 1\]In\'equation num\'ero 3 \[\left(10 x + 2\right)^{2} \leq \left(10 x + 7\right)^{2}\]In\'equation num\'ero 4 \[\left(6 x + 2\right)^{2} \geq 2 x + 6\]In\'equation num\'ero 5 \[4 x^{2} + 4 x + 9 \leq 5 x^{2} + 3 x + 2\]In\'equation num\'ero 6 \[7 x + 7 \geq \left(6 x + 2\right)^{2}\]In\'equation num\'ero 7 \[\left(10 x + 2\right)^{2} \leq \left(10 x + 9\right)^{2}\]In\'equation num\'ero 8 \[8 x^{2} + 7 \leq 3 x + 5\]In\'equation num\'ero 9 \[x + 3 \leq \left(5 x + 3\right)^{2}\]In\'equation num\'ero 10 \[\left(6 x + 3\right)^{2} \leq \left(9 x + 8\right)^{2}\]In\'equation num\'ero 11 \[\left(10 x + 5\right)^{2} \leq 5 x^{2} + x\]In\'equation num\'ero 12 \[\left(6 x + 6\right)^{2} \leq x + 3\]In\'equation num\'ero 13 \[8 x + 1 \geq 10 x + 1\]In\'equation num\'ero 14 \[\left(9 x + 7\right)^{2} \leq \left(3 x + 3\right)^{2}\]In\'equation num\'ero 15 \[\left(9 x + 1\right)^{2} \leq 3 x^{2} + 4 x + 2\]In\'equation num\'ero 16 \[x + 2 \geq 5 x + 1\]In\'equation num\'ero 17 \[\left(2 x + 2\right)^{2} \leq 8 x + 8\]In\'equation num\'ero 18 \[\left(5 x + 2\right)^{2} \leq \left(2 x + 3\right)^{2}\]In\'equation num\'ero 19 \[\left(6 x + 7\right)^{2} \geq 3 x\]
 \section{Exercices de In\'equation Moyens}

 R\'esoudre les in\'equations suivantes : 
In\'equation num\'ero 0 \[\frac{x + 7}{3 x + 4} \leq \frac{3 \sqrt{10} \pi}{10}\]In\'equation num\'ero 1 \[\frac{4 x + 6}{5 x + 6} \geq \frac{\sqrt{130}}{10}\]In\'equation num\'ero 2 \[\frac{5 x + 2}{3 x + 6} \geq \frac{2 \sqrt{5}}{\pi}\]In\'equation num\'ero 3 \[\frac{10 x + 10}{8 x + 3} \leq \sqrt{7}\]In\'equation num\'ero 4 \[\frac{x + 2}{x + 5} \leq \frac{\sqrt{70}}{7}\]In\'equation num\'ero 5 \[\frac{4 x + 7}{x + 5} \leq \frac{\sqrt{6}}{6 \pi}\]In\'equation num\'ero 6 \[\frac{9 x + 6}{9 x + 7} \leq \sqrt{10}\]In\'equation num\'ero 7 \[\frac{2 x + 4}{8 x + 10} \leq \frac{\sqrt{165}}{11 \pi}\]In\'equation num\'ero 8 \[\frac{8 x + 10}{9 x + 4} \geq \frac{1}{2}\]In\'equation num\'ero 9 \[\frac{2 x + 2}{x + 9} \leq \frac{\sqrt{143}}{13 \pi}\]In\'equation num\'ero 10 \[\frac{3 x + 8}{10 x + 8} \leq \frac{\sqrt{2}}{6 \pi}\]In\'equation num\'ero 11 \[\frac{8 x + 8}{8 x + 9} \geq \sqrt{6}\]In\'equation num\'ero 12 \[\frac{10 x + 9}{8 x + 1} \leq \frac{4 \sqrt{19}}{19}\]In\'equation num\'ero 13 \[\frac{5 x + 1}{3 x + 6} \geq \frac{\sqrt{55} \pi}{5}\]In\'equation num\'ero 14 \[\frac{7 x + 1}{10 x + 9} \leq \frac{\sqrt{2}}{\pi}\]In\'equation num\'ero 15 \[\frac{7 x + 4}{2 x + 6} \geq \frac{4}{\pi}\]In\'equation num\'ero 16 \[\frac{3 x + 3}{7 x + 2} \leq \frac{\sqrt{2}}{2 \pi}\]In\'equation num\'ero 17 \[\frac{2 x + 7}{3 x + 5} \leq \sqrt{2}\]In\'equation num\'ero 18 \[\frac{x + 2}{9 x + 7} \geq \frac{\sqrt{10}}{10 \pi}\]In\'equation num\'ero 19 \[\frac{10 x + 10}{10 x + 7} \leq 4\]In\'equation num\'ero 20 \[\frac{9 x + 5}{2 x + 8} \geq \frac{\sqrt{2}}{4 \pi}\]In\'equation num\'ero 21 \[\frac{3 x + 1}{5 x + 6} \geq \sqrt{13}\]In\'equation num\'ero 22 \[\frac{8 x + 10}{2 x + 1} \geq \frac{\sqrt{2} \pi}{2}\]In\'equation num\'ero 23 \[\frac{2 x + 3}{7 x + 9} \geq 3 \sqrt{2} \pi\]In\'equation num\'ero 24 \[\frac{10 x + 9}{4 x + 8} \geq 2 \sqrt{3}\]In\'equation num\'ero 25 \[\frac{4 x + 8}{5 x + 7} \leq \frac{\sqrt{15} \pi}{3}\]In\'equation num\'ero 26 \[\frac{6 x + 7}{5 x + 9} \geq \frac{\sqrt{5}}{10}\]In\'equation num\'ero 27 \[\frac{6 x + 4}{x + 6} \geq \frac{\sqrt{42}}{7 \pi}\]In\'equation num\'ero 28 \[\frac{2 x + 2}{6 x + 6} \geq \sqrt{3} \pi\]In\'equation num\'ero 29 \[\frac{10 x + 7}{3 x + 2} \geq \frac{\sqrt{10}}{2}\]In\'equation num\'ero 30 \[\frac{5 x + 4}{10 x + 5} \leq \frac{1}{2}\]In\'equation num\'ero 31 \[\frac{7 x + 7}{x + 2} \geq \frac{\sqrt{2} \pi}{2}\]In\'equation num\'ero 32 \[\frac{9 x + 5}{2 x + 1} \leq \frac{\sqrt{7}}{3}\]In\'equation num\'ero 33 \[\frac{x + 1}{10 x + 9} \geq \frac{\sqrt{95}}{19 \pi}\]In\'equation num\'ero 34 \[\frac{10 x + 8}{x + 10} \leq \sqrt{14}\]In\'equation num\'ero 35 \[\frac{2 x + 3}{2 x + 9} \leq \frac{\sqrt{2}}{6}\]In\'equation num\'ero 36 \[\frac{10 x + 4}{7 x + 8} \leq \frac{\sqrt{55}}{5 \pi}\]In\'equation num\'ero 37 \[\frac{2 x + 8}{4 x + 10} \leq \frac{\sqrt{21}}{3 \pi}\]In\'equation num\'ero 38 \[\frac{3 x + 2}{2 x + 9} \leq \frac{3 \sqrt{13}}{13}\]In\'equation num\'ero 39 \[\frac{x + 1}{9 x + 5} \leq \frac{\sqrt{5}}{10}\]
 \section{Exercices de In\'equation Durs}

 R\'esoudre les in\'equations suivantes : 
In\'equation num\'ero 0 \[\frac{x + 1}{3 x + 1} \geq \frac{4 x + 1}{6 x + 9}\]In\'equation num\'ero 1 \[\frac{16 x^{2} - 40 x + 25}{8 x + 10} \leq \frac{4 x^{2} + 4 x + 1}{8 x + 10}\]In\'equation num\'ero 2 \[\frac{4 x^{2} + 4 x + 1}{5 x + 2} \leq \frac{9 x^{2} - 6 x + 1}{5 x + 2}\]In\'equation num\'ero 3 \[\frac{4 x^{2} + 16 x + 16}{7 x + 1} \geq \frac{16 x^{2} - 8 x + 1}{7 x + 1}\]In\'equation num\'ero 4 \[\frac{10 x + 2}{10 x + 5} \leq \frac{4 x + 4}{6 x + 5}\]In\'equation num\'ero 5 \[\frac{x^{2} + 2 x + 1}{5 x + 8} \leq \frac{16 x^{2} - 24 x + 9}{5 x + 8}\]In\'equation num\'ero 6 \[\frac{4 x^{2} - 4 x + 1}{8 x + 3} \geq \frac{x^{2} - 8 x + 16}{8 x + 3}\]In\'equation num\'ero 7 \[\frac{2 x + 2}{2 x + 5} \geq \frac{x + 3}{x + 6}\]In\'equation num\'ero 8 \[\frac{5 x + 9}{9 x + 9} \geq \frac{5 x + 1}{7 x + 3}\]In\'equation num\'ero 9 \[\frac{4 x + 6}{2 x + 3} \leq \frac{2 x + 3}{7 x + 2}\]In\'equation num\'ero 10 \[\frac{6 x + 3}{7 x + 2} \geq \frac{7 x + 3}{6 x + 5}\]In\'equation num\'ero 11 \[\frac{4 x^{2} - 12 x + 9}{6 x + 5} \geq \frac{x^{2} - 6 x + 9}{6 x + 5}\]In\'equation num\'ero 12 \[\frac{x + 6}{x + 2} \leq \frac{10 x + 5}{4 x + 7}\]In\'equation num\'ero 13 \[\frac{2 x + 9}{5 x + 7} \geq \frac{3 x + 5}{6 x + 10}\]In\'equation num\'ero 14 \[\frac{7 x + 10}{2 x + 7} \leq \frac{4 x + 7}{10 x + 3}\]In\'equation num\'ero 15 \[\frac{4 x + 3}{2 x + 10} \leq \frac{4 x + 9}{3 x + 9}\]In\'equation num\'ero 16 \[\frac{x^{2} - 10 x + 25}{10 x + 7} \leq \frac{16 x^{2} - 40 x + 25}{10 x + 7}\]In\'equation num\'ero 17 \[\frac{9 x^{2} - 18 x + 9}{10 x + 7} \geq \frac{9 x^{2} + 18 x + 9}{10 x + 7}\]In\'equation num\'ero 18 \[\frac{x^{2} - 2 x + 1}{2 x + 10} \geq \frac{25 x^{2} - 50 x + 25}{2 x + 10}\]In\'equation num\'ero 19 \[\frac{x^{2} - 4 x + 4}{9 x + 8} \geq \frac{9 x^{2} - 24 x + 16}{9 x + 8}\]In\'equation num\'ero 20 \[\frac{4 x^{2} - 8 x + 4}{8 x + 3} \leq \frac{25 x^{2} - 20 x + 4}{8 x + 3}\]In\'equation num\'ero 21 \[\frac{9 x + 8}{x + 7} \geq \frac{9 x + 3}{4 x + 3}\]In\'equation num\'ero 22 \[\frac{9 x + 6}{4 x + 1} \geq \frac{3 x + 7}{2 x + 9}\]In\'equation num\'ero 23 \[\frac{10 x + 7}{10 x + 9} \leq \frac{2 x + 6}{9 x + 7}\]In\'equation num\'ero 24 \[\frac{5 x + 4}{x + 9} \geq \frac{2 x + 3}{7 x + 3}\]In\'equation num\'ero 25 \[\frac{16 x^{2} - 8 x + 1}{4 x + 5} \geq \frac{16 x^{2} + 24 x + 9}{4 x + 5}\]In\'equation num\'ero 26 \[\frac{4 x^{2} - 12 x + 9}{6 x + 7} \geq \frac{x^{2} - 10 x + 25}{6 x + 7}\]In\'equation num\'ero 27 \[\frac{25 x^{2} - 40 x + 16}{4 x + 3} \leq \frac{25 x^{2} + 10 x + 1}{4 x + 3}\]In\'equation num\'ero 28 \[\frac{5 x + 4}{10 x + 3} \geq \frac{x + 9}{7 x + 4}\]In\'equation num\'ero 29 \[\frac{10 x + 9}{7 x + 5} \geq \frac{6 x + 1}{10 x + 7}\]In\'equation num\'ero 30 \[\frac{9 x^{2} - 12 x + 4}{6 x + 5} \leq \frac{x^{2} - 10 x + 25}{6 x + 5}\]In\'equation num\'ero 31 \[\frac{9 x^{2} + 6 x + 1}{9 x + 7} \leq \frac{25 x^{2} - 50 x + 25}{9 x + 7}\]In\'equation num\'ero 32 \[\frac{x^{2} - 10 x + 25}{3 x + 4} \leq \frac{4 x^{2} - 20 x + 25}{3 x + 4}\]In\'equation num\'ero 33 \[\frac{6 x + 1}{4 x + 5} \geq \frac{6 x + 1}{8 x + 10}\]In\'equation num\'ero 34 \[\frac{9 x + 4}{x + 5} \leq \frac{2 x + 9}{9 x + 1}\]In\'equation num\'ero 35 \[\frac{16 x^{2} - 16 x + 4}{6 x + 8} \geq \frac{9 x^{2} - 30 x + 25}{6 x + 8}\]In\'equation num\'ero 36 \[\frac{6 x + 3}{9 x + 4} \geq \frac{3 x + 4}{2 x + 10}\]In\'equation num\'ero 37 \[\frac{2 x + 7}{x + 10} \geq \frac{7 x + 1}{9 x + 4}\]In\'equation num\'ero 38 \[\frac{8 x + 5}{7 x + 4} \leq \frac{4 x + 7}{10 x + 10}\]In\'equation num\'ero 39 \[\frac{9 x^{2} - 18 x + 9}{2 x + 8} \geq \frac{25 x^{2} - 10 x + 1}{2 x + 8}\]
 \section{Exercices de Tableaux de Variation Faciles}

 Donner les variations des fonctions suivantes : 
Tableaux de Variation num\'ero 0 \[f(x) = 3 x^{2} + \sqrt{3} x + 2 x + 1 + \sqrt{7} \pi\]Tableaux de Variation num\'ero 1 \[f(x) = x^{2} + \sqrt{15} x + 5 x + \sqrt{6} + 8\]Tableaux de Variation num\'ero 2 \[f(x) = \left(4 \sqrt{17} x + 6 \pi\right)^{3}\]Tableaux de Variation num\'ero 3 \[f(x) = \sqrt{14} \pi x^{2} + \sqrt{6} \pi x + 8 x + \pi + 7\]Tableaux de Variation num\'ero 4 \[f(x) = \left(5 x + 6 \sqrt{14} \pi\right)^{3}\]Tableaux de Variation num\'ero 5 \[f(x) = 3 \sqrt{10} \pi x^{3} + 12 \sqrt{2} \pi x^{2} + 2 \sqrt{17} x + 9 \pi\]Tableaux de Variation num\'ero 6 \[f(x) = \sqrt{15} x^{2} + 6 x + \sqrt{10} \pi x + 1 + \sqrt{6}\]Tableaux de Variation num\'ero 7 \[f(x) = \left(28 x + 21 \sqrt{2}\right)^{3}\]Tableaux de Variation num\'ero 8 \[f(x) = 7 \sqrt{19} x^{3} + 4 \sqrt{14} x^{2} + 6 \sqrt{17} x + 18 \sqrt{5}\]Tableaux de Variation num\'ero 9 \[f(x) = \left(3 x + 4\right) \left(6 x + 1\right)\]Tableaux de Variation num\'ero 10 \[f(x) = \sqrt{10} x + \sqrt{13} x + 8 x + 5 + 4 \pi\]Tableaux de Variation num\'ero 11 \[f(x) = 3 \pi x^{2} + 6 x + \sqrt{5} \pi x + \sqrt{17} \pi\]Tableaux de Variation num\'ero 12 \[f(x) = \left(9 x + 3\right) \left(9 x + 5\right)^{2}\]Tableaux de Variation num\'ero 13 \[f(x) = \pi x + 8 x + \sqrt{10} \pi x + 8\]Tableaux de Variation num\'ero 14 \[f(x) = \left(9 x + 9 \pi\right)^{3}\]Tableaux de Variation num\'ero 15 \[f(x) = \left(3 x + 8\right) \left(6 x + 5\right)\]Tableaux de Variation num\'ero 16 \[f(x) = \left(x + 3\right) \left(6 x^{2} + 3 x + 1\right)\]Tableaux de Variation num\'ero 17 \[f(x) = 7 \sqrt{2} x^{3} + 6 \sqrt{3} \pi x^{2} + 3 \sqrt{2} x + 9 \sqrt{17} \pi\]Tableaux de Variation num\'ero 18 \[f(x) = 2 \sqrt{3} x^{2} + 3 x + \sqrt{2} \pi x + 2 + 2 \sqrt{5}\]Tableaux de Variation num\'ero 19 \[f(x) = \sqrt{2} \pi x^{2} + 5 x + \sqrt{17} \pi x + 4 + 3 \pi\]
 \section{Exercices de Tableaux de Variation Moyens}

 Donner les variations des fonctions suivantes : 
Tableaux de Variation num\'ero 0 \[f(x) = \sqrt{x + 8} \cdot \left(3 x + 3\right)\]Tableaux de Variation num\'ero 1 \[f(x) = \frac{\sqrt{3} \pi x + \sqrt{6} \pi}{\sqrt{19} x + 2 \sqrt{3} \pi}\]Tableaux de Variation num\'ero 2 \[f(x) = \left(2 x + 3\right) \sqrt{6 x + 3}\]Tableaux de Variation num\'ero 3 \[f(x) = \frac{2 x + \sqrt{10}}{x + 3 \sqrt{2}}\]Tableaux de Variation num\'ero 4 \[f(x) = \left(4 x + 5\right) \sqrt{4 x + 7}\]Tableaux de Variation num\'ero 5 \[f(x) = \sqrt{3 x + 6} \cdot \left(6 x + 5\right)\]Tableaux de Variation num\'ero 6 \[f(x) = \sqrt{5 x + 4} \cdot \left(6 x + 9\right)\]Tableaux de Variation num\'ero 7 \[f(x) = \sqrt{6 x + 1} \cdot \left(10 x + 3\right)\]Tableaux de Variation num\'ero 8 \[f(x) = \frac{\sqrt{11} \pi x + \sqrt{5} \pi}{\sqrt{14} \pi x + \sqrt{2}}\]Tableaux de Variation num\'ero 9 \[f(x) = \frac{\sqrt{15} x + 2}{2 \sqrt{2} x + \sqrt{6} \pi}\]Tableaux de Variation num\'ero 10 \[f(x) = \frac{\sqrt{14} \pi x + \sqrt{10}}{\sqrt{13} x + \sqrt{2}}\]Tableaux de Variation num\'ero 11 \[f(x) = \left(4 x + 4\right) \sqrt{4 x + 8}\]Tableaux de Variation num\'ero 12 \[f(x) = \frac{\sqrt{3} x + \sqrt{11} \pi}{\pi x + \sqrt{13} \pi}\]Tableaux de Variation num\'ero 13 \[f(x) = \frac{2 x + 2 \sqrt{2} \pi}{3 \sqrt{2} \pi x + \sqrt{2} \pi}\]Tableaux de Variation num\'ero 14 \[f(x) = \frac{\sqrt{17} x + 2 \sqrt{5} \pi}{3 \sqrt{2} x + \sqrt{2} \pi}\]Tableaux de Variation num\'ero 15 \[f(x) = \sqrt{3 x + 2} \cdot \left(9 x + 2\right)\]Tableaux de Variation num\'ero 16 \[f(x) = \left(3 x + 8\right) \sqrt{10 x + 2}\]Tableaux de Variation num\'ero 17 \[f(x) = \frac{\sqrt{17} \pi x + 2}{x + \sqrt{6}}\]Tableaux de Variation num\'ero 18 \[f(x) = \frac{\sqrt{7} x + 2 \sqrt{2}}{\pi x + \sqrt{3}}\]Tableaux de Variation num\'ero 19 \[f(x) = \sqrt{6 x + 9} \cdot \left(9 x + 7\right)\]Tableaux de Variation num\'ero 20 \[f(x) = \sqrt{6 x + 6} \cdot \left(10 x + 5\right)\]Tableaux de Variation num\'ero 21 \[f(x) = \left(5 x + 1\right) \sqrt{9 x + 4}\]Tableaux de Variation num\'ero 22 \[f(x) = \sqrt{8 x + 9} \cdot \left(10 x + 1\right)\]Tableaux de Variation num\'ero 23 \[f(x) = \sqrt{x + 9} \cdot \left(7 x + 6\right)\]Tableaux de Variation num\'ero 24 \[f(x) = \left(6 x + 3\right) \sqrt{10 x + 8}\]Tableaux de Variation num\'ero 25 \[f(x) = \frac{3 x + \sqrt{15} \pi}{\sqrt{15} \pi x + \sqrt{15} \pi}\]Tableaux de Variation num\'ero 26 \[f(x) = \sqrt{5 x + 2} \cdot \left(9 x + 2\right)\]Tableaux de Variation num\'ero 27 \[f(x) = \left(8 x + 3\right) \sqrt{8 x + 4}\]Tableaux de Variation num\'ero 28 \[f(x) = \sqrt{x + 8} \cdot \left(6 x + 1\right)\]Tableaux de Variation num\'ero 29 \[f(x) = \sqrt{6 x + 9} \cdot \left(8 x + 4\right)\]Tableaux de Variation num\'ero 30 \[f(x) = \left(6 x + 4\right) \sqrt{6 x + 8}\]Tableaux de Variation num\'ero 31 \[f(x) = \left(3 x + 2\right) \sqrt{4 x + 8}\]Tableaux de Variation num\'ero 32 \[f(x) = \left(x + 5\right) \sqrt{4 x + 1}\]Tableaux de Variation num\'ero 33 \[f(x) = \frac{2 \sqrt{3} x + 3 \pi}{\pi x + 2 \sqrt{2} \pi}\]Tableaux de Variation num\'ero 34 \[f(x) = \sqrt{6 x + 10} \cdot \left(8 x + 2\right)\]Tableaux de Variation num\'ero 35 \[f(x) = \frac{\sqrt{17} x + \sqrt{3}}{3 \sqrt{2} x + 1}\]Tableaux de Variation num\'ero 36 \[f(x) = \left(8 x + 8\right) \sqrt{10 x + 7}\]Tableaux de Variation num\'ero 37 \[f(x) = \frac{\pi x + \pi}{\sqrt{3} \pi x + 1}\]Tableaux de Variation num\'ero 38 \[f(x) = \frac{4 x + \pi}{4 \pi x + \sqrt{2} \pi}\]Tableaux de Variation num\'ero 39 \[f(x) = \frac{2 \sqrt{5} \pi x + 1}{\sqrt{2} x + \sqrt{19}}\]
 \section{Exercices de Tableaux de Variation Durs}

 Donner les variations des fonctions suivantes : 
Tableaux de Variation num\'ero 0 \[f(x) = \frac{\sqrt{7} \pi \sqrt{x + 2 \pi}}{\left(10 x + 5\right)^{2}}\]Tableaux de Variation num\'ero 1 \[f(x) = \frac{\sqrt{14} \pi \sqrt{\sqrt{17} \pi x + 1}}{x + 9}\]Tableaux de Variation num\'ero 2 \[f(x) = \frac{\sqrt{15} \pi \sqrt{\pi x + 1}}{4}\]Tableaux de Variation num\'ero 3 \[f(x) = 3 \sqrt{2} \pi \left(2 \sqrt{3} x + 4 \pi\right) \sqrt{\left(10 x + 5\right)^{2}}\]Tableaux de Variation num\'ero 4 \[f(x) = \sqrt{15} \cdot \left(4 \pi x + \sqrt{2}\right) \sqrt{\left(x + 3\right)^{2}}\]Tableaux de Variation num\'ero 5 \[f(x) = 4 \pi \left(x + \sqrt{6} \pi\right) \sqrt{5 x^{2} + 9 x + 5}\]Tableaux de Variation num\'ero 6 \[f(x) = \frac{2 \sqrt{2} \sqrt{x + \sqrt{5} \pi}}{\left(10 x + 8\right)^{2}}\]Tableaux de Variation num\'ero 7 \[f(x) = \frac{\sqrt{5} \sqrt{2 x + \pi}}{x^{2} + 5 x + 2}\]Tableaux de Variation num\'ero 8 \[f(x) = \sqrt{17} \left(\sqrt{3} x + \pi\right) \sqrt{\left(3 x + 3\right)^{2}}\]Tableaux de Variation num\'ero 9 \[f(x) = \frac{2 \sqrt{2 x + 4 \pi}}{4 x^{2} + 6 x + 3}\]Tableaux de Variation num\'ero 10 \[f(x) = \frac{\sqrt{11} \pi \sqrt{\sqrt{19} x + \pi}}{\left(9 x + 7\right)^{2}}\]Tableaux de Variation num\'ero 11 \[f(x) = \frac{\sqrt{10} \sqrt{\sqrt{19} x + 2 \sqrt{5}}}{\left(2 x + 2\right)^{2}}\]Tableaux de Variation num\'ero 12 \[f(x) = \sqrt{19} \pi \left(\sqrt{17} x + 4\right) \sqrt{\left(4 x + 7\right)^{2}}\]Tableaux de Variation num\'ero 13 \[f(x) = \frac{4 \pi \sqrt{3 \sqrt{2} \pi x + \sqrt{15} \pi}}{\left(6 x + 2\right)^{2}}\]Tableaux de Variation num\'ero 14 \[f(x) = \frac{\sqrt{11} \sqrt{x + \sqrt{14} \pi}}{\left(4 x + 8\right)^{2}}\]Tableaux de Variation num\'ero 15 \[f(x) = 3 \sqrt{4 x^{2} + 4 x} \left(3 \sqrt{2} \pi x + \sqrt{17} \pi\right)\]Tableaux de Variation num\'ero 16 \[f(x) = 2 \sqrt{3} \pi \left(\pi x + \sqrt{14} \pi\right) \sqrt{\left(2 x + 7\right)^{2}}\]Tableaux de Variation num\'ero 17 \[f(x) = 3 \sqrt{2} \sqrt{2 x + 8} \left(\sqrt{10} x + 4\right)\]Tableaux de Variation num\'ero 18 \[f(x) = \sqrt{6} \pi \left(\sqrt{5} x + 2 \sqrt{3} \pi\right) \sqrt{\left(2 x + 3\right)^{2}}\]Tableaux de Variation num\'ero 19 \[f(x) = \frac{\sqrt{17} \pi \sqrt{2 x + \sqrt{11}}}{\left(3 x + 8\right)^{2}}\]Tableaux de Variation num\'ero 20 \[f(x) = \frac{4 \pi \sqrt{\sqrt{15} \pi x + \sqrt{19} \pi}}{\left(9 x + 1\right)^{2}}\]Tableaux de Variation num\'ero 21 \[f(x) = \pi \left(3 \sqrt{2} x + \sqrt{15} \pi\right) \sqrt{\left(10 x + 1\right)^{2}}\]Tableaux de Variation num\'ero 22 \[f(x) = 2 \sqrt{3} \pi \sqrt{x + 3} \cdot \left(3 x + \pi\right)\]Tableaux de Variation num\'ero 23 \[f(x) = \sqrt{17} \left(x + \sqrt{14} \pi\right) \sqrt{x^{2} + 2 x + 3}\]Tableaux de Variation num\'ero 24 \[f(x) = 4 \sqrt{2 x + 2} \left(\pi x + \pi\right)\]Tableaux de Variation num\'ero 25 \[f(x) = \frac{\sqrt{11} \pi \sqrt{\sqrt{6} \pi x + \sqrt{6}}}{x + 3}\]Tableaux de Variation num\'ero 26 \[f(x) = \pi \left(4 \pi x + \sqrt{2} \pi\right) \sqrt{\left(4 x + 4\right)^{2}}\]Tableaux de Variation num\'ero 27 \[f(x) = \frac{\sqrt{\sqrt{10} \pi x + \sqrt{6}}}{\left(4 x + 8\right)^{2}}\]Tableaux de Variation num\'ero 28 \[f(x) = \sqrt{7} \pi \left(\sqrt{15} \pi x + \sqrt{5}\right) \sqrt{\left(6 x + 4\right)^{2}}\]Tableaux de Variation num\'ero 29 \[f(x) = 2 \sqrt{2} \left(\sqrt{7} \pi x + \pi\right) \sqrt{9 x^{2} + 8 x + 5}\]Tableaux de Variation num\'ero 30 \[f(x) = \frac{3 \sqrt{2} \sqrt{\pi x + 1}}{\left(2 x + 3\right)^{2}}\]Tableaux de Variation num\'ero 31 \[f(x) = \frac{\sqrt{13} \pi \sqrt{3 \sqrt{2} \pi x + \sqrt{13} \pi}}{\left(x + 10\right)^{2}}\]Tableaux de Variation num\'ero 32 \[f(x) = \left(2 \sqrt{2} \pi x + 3 \sqrt{2} \pi\right) \sqrt{\left(9 x + 2\right)^{2}}\]Tableaux de Variation num\'ero 33 \[f(x) = \sqrt{5} \pi \sqrt{10 x + 7} \cdot \left(3 \pi x + \sqrt{3} \pi\right)\]Tableaux de Variation num\'ero 34 \[f(x) = \frac{2 \sqrt{\sqrt{14} \pi x + 1}}{x^{2} + 2 x + 1}\]Tableaux de Variation num\'ero 35 \[f(x) = \sqrt{19} \sqrt{9 x + 7} \left(\sqrt{6} x + 2 \sqrt{3}\right)\]Tableaux de Variation num\'ero 36 \[f(x) = \left(\sqrt{19} x + \pi\right) \sqrt{\left(9 x + 1\right)^{2}}\]Tableaux de Variation num\'ero 37 \[f(x) = \frac{\sqrt{15} \pi \sqrt{\sqrt{10} \pi x + \sqrt{6}}}{\left(9 x + 6\right)^{2}}\]Tableaux de Variation num\'ero 38 \[f(x) = \frac{\sqrt{19} \pi \sqrt{2 \sqrt{3} x + \sqrt{7} \pi}}{5 x^{2} + x + 7}\]Tableaux de Variation num\'ero 39 \[f(x) = \frac{\sqrt{17} \sqrt{\sqrt{2} x + 1}}{\left(10 x + 3\right)^{2}}\]\end{document}